%% =====================================
%% FIGURE 1
%% =====================================
%% Description: Integrate a figure
%% Can be compiled independently or as part of main.tex

\documentclass[../../main.tex]{subfiles}

%% Begin elements/figures/fig01.tex
    \begin{document}

\begin{figure}[ht]
    \centering
    %%
\begin{tikzpicture}[
    %% General parameters
    %%
    node distance=0.4cm and 1.8cm,
    %%
    %% Node style
    %%
    root/.style={
        rectangle,
        draw=none, fill=gray!15,
        minimum height=1cm, minimum width=3cm,
        font=\sffamily\bfseries,
        inner sep=6pt,
        text=customizedcolor!100
    },
    folder/.style={
        rectangle,
        draw=none, fill=gray!5,
        minimum height=0.65cm, minimum width=2.8cm,
        font=\sffamily\small,
        inner sep=5pt,
        text=black!70
    },
    file/.style={
        rectangle,
        draw=none, fill=white,
        minimum height=0.5cm, minimum width=2.5cm,
        font=\ttfamily\footnotesize,
        inner sep=4pt,
        text=black!60
    },
    %%
    %% small boxes
    see/.style={
        rectangle,
        draw=none,
        fill=gray!0,
        minimum height=0.65cm,
        font=\sffamily\scriptsize,
        inner sep=4pt,
        text=black!65,
        align=left
    },
    %%
    %% Line style
    line/.style={
        line width=0.1pt,
        gray!15
    }
]
%%
%% Branch 2: Main files
%%
\node[file] (main) {main.tex};
\node[file, below=of main] (latexmk) {latexmkrc};
%%
%% Branch 2: Main folders
%%
\node[folder, below=0.8cm of latexmk] (content) {content/};
%%
\node[folder, below=0.6cm of content] (elements) {elements/};
%%
\node[folder, below=0.6cm of elements] (asset) {asset/};
%%
\node[folder, below=0.6cm of asset] (extra) {extra/};
%%
%% Compute vertical center
%%
\coordinate (rightcenter) at ($(main)!0.5!(extra)$);
%%
%% Root node
\node[root, align=center] (root) at ($(rightcenter) + (-5cm,0)$) {PhD Thesis\\Template};
%%
%% Connections
%%
\draw[line] (root.east) -- (main.west);
%%
\draw[line] (root.east) -- (latexmk.west);
%%
\draw[line] (root.east) -- (content.west);
%%
\draw[line] (root.east) -- (elements.west);
%%
\draw[line] (root.east) -- (asset.west);
%%
\draw[line] (root.east) -- (extra.west);
%%
%% See Fig
%%
\node[see, right=0.6cm of content] (seecontent)
  {see \hyperref[fig:appendices-D1]{Figure~\ref{fig:appendices-D1}},
   \hyperref[fig:appendices-D1]{p.}~\pageref{fig:appendices-D1}};
   %%
\node[see, right=0.6cm of elements] (seeelements)
  {see \hyperref[fig:appendices-E1]{Figure~\ref{fig:appendices-E1}},
   \hyperref[fig:appendices-E1]{p.}~\pageref{fig:appendices-E1}};
   %%
\node[see, right=0.6cm of asset] (seeasset)
  {see \hyperref[fig:appendices-F1]{Figure~\ref{fig:appendices-F1}},
   \hyperref[fig:appendices-F1]{p.}~\pageref{fig:appendices-F1}};
   %%
\node[see, right=0.6cm of extra] (seeextra)
  {see \hyperref[fig:appendices-G1]{Figure~\ref{fig:appendices-G1}},
   \hyperref[fig:appendices-G1]{p.}~\pageref{fig:appendices-G1}};
   %%
\end{tikzpicture}
    %%
    \vspace{0.5cm}
    %%
    \caption{General Structure of the \latexword{LaTeX} Thesis Template}
    \label{fig:introduction-01}
    %%
\end{figure}

%% End elements/figures/fig01.tex
    \end{document}