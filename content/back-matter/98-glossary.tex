%% =====================================
%% GLOSSARY
%% =====================================
%% Description: Automatic glossary
%% Can be compiled independently or as part of main.tex

\documentclass[../../main.tex]{subfiles}

%% Begin content/back-matter/98-glossary.tex
    \begin{document}

%% =====================================
%% DEFINITIONS
%% =====================================

%% Notion 1
\newglossaryentry{notion1}{
    name={First notion:},
    text={first notion},
    description={
    \footnotesize{
\enquote{\loremmedium}~\autocite[5]{example2022}\index{Casual, Name|pagebf}
        }
    }}

%% Notion 2
\newglossaryentry{notion2}{
    name={Second notion:},
    text={second notion},
    description={
    \footnotesize{
\loremlong
        }
    }}

%% =====================================
%% PRINT GLOSSARY
%% =====================================

    \cleardoublepage
    \markright{Glossary}{}
    %\phantomsection
    \addcontentsline{toc}{part}{Glossary}
    \label{sec:glossary}

%% Print glossary
\printglossary[
  title=Glossary,
  style=long
]

%% Sub-bibliography for glossary
    \newpage
    \begingroup
    \renewcommand{\bibfont}{\scriptsize}
\printbibliography[
  segment=\therefsegment,
  heading=subbibintoc,
  title={Glossary References},
  label=glossary:bibliography
]
    \endgroup

%% End content/back-matter/98-glossary.tex
    \end{document}