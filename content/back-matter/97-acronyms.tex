%% =====================================
%% ACRONYMS
%% =====================================
%% Description: Automatic acronyms
%% Can be compiled independently or as part of main.tex

\documentclass[../../main.tex]{subfiles}

%% Begin content/back-matter/97-acronyms.tex
    \begin{document}

%% =====================================
%% DEFINITIONS
%% =====================================

%% PhD acronym
\newacronym[description={Doctor of Philosophy.}]{PhD}{PhD}{acronym1}

%% First acronym
\newacronym[description={This is an acronym.}]{A1}{A1}{acronym1}

%% Second acronym
\newacronym[description={This is another acronym.}]{A2}{A2}{acronym2}

%% Third acronym
\newacronym[description={And a third acronym.}]{B1}{B1}{acronym3}

%% =====================================
%% PRINT ACRONYMS
%% =====================================

    \cleardoublepage
    \markright{Acronyms}{}
    %\phantomsection
    \addcontentsline{toc}{part}{List of Acronyms}
    \label{part:acronyms}

%% Make acronyms bold in the list
    \renewcommand{\glsnamefont}[1]{\textbf{#1}}

%% Print acronym list
\printglossary[
  type=\acronymtype,
  title=Abbreviations and Acronyms,
  style=long
]

%% End content/back-matter/97-acronyms.tex
    \end{document}